%--------------字体------------------
\newcommand{\FakeBoldSize}{4}
% Windows or other platform
% \setCJKfamilyfont{zhfs}   {FangSong} [AutoFakeBold={\FakeBoldSize}]
% \setCJKfamilyfont{zhhei}  {SimHei}   [AutoFakeBold={\FakeBoldSize}]
% \setCJKfamilyfont{zhkai}  {KaiTi}    [AutoFakeBold={\FakeBoldSize}]
% \setCJKfamilyfont{zhsong} {SimSun}   [AutoFakeBold={\FakeBoldSize}]
% \setCJKmainfont[AutoFakeBold={\FakeBoldSize}]{SimSun}
%\setmainfont{Times New Roman}
% \renewcommand{\bibfont}{\zihao{5}\songti}

%%%% 定义字体族简称
\setCJKfamilyfont{song}{SimSun}[AutoFakeBold={\FakeBoldSize}]
\setCJKfamilyfont{hei}{SimHei}[AutoFakeBold={\FakeBoldSize}]
\setCJKfamilyfont{kai}{KaiTi}[AutoFakeBold={\FakeBoldSize}]
\setCJKfamilyfont{lishu}{LiSu}[AutoFakeBold={\FakeBoldSize}]
\setCJKfamilyfont{fangsong}{FangSong}[AutoFakeBold={\FakeBoldSize}]

%%%% 设置全文中英文主字体
\setCJKmainfont[AutoFakeBold={\FakeBoldSize}]{SimSun} % 中文主字体
\setmainfont{Times New Roman} % 英文主字体

%%%% 创建字体格式设置命令
% 伪粗体
\newcommand{\fakehei}{\CJKfamily{hei}}
\newcommand{\fakesong}{\CJKfamily{song}}
\newcommand{\fakeli}{\CJKfamily{lishu}}
% 正常体
\renewcommand{\heiti}{\CJKfamily{SimHei}}
\renewcommand{\songti}{\CJKfamily{SimSun}}
\newcommand{\lishu}{\CJKfamily{LiSu}}
\newcommand{\sanhao}{\fontsize{15.75pt}{\baselineskip}\selectfont}
\newcommand{\sihao}{\fontsize{14pt}{\baselineskip}\selectfont}
\newcommand{\xiaosi}{\fontsize{12pt}{\baselineskip}\selectfont}
\newcommand{\wuhao}{\fontsize{10.5pt}{\baselineskip}\selectfont}
\newcommand{\xiaowu}{\fontsize{9pt}{\baselineskip}\selectfont}
\newcommand{\timu}{\fontsize{16pt}{\baselineskip}\selectfont\CJKfamily{hei}}%题目格式
\newcommand{\zhaiyao}{\fontsize{16pt}{\baselineskip}\selectfont\CJKfamily{li}}%摘要格式

%%%% 目录字体
\titlecontents{chapter}[0pt]{\heiti\zihao{-4}}{\thecontentslabel\quad}{}
        {\titlerule*[10pt]{$\ldots$}\contentspage}

\titlecontents{section}[0pt]{\songti\zihao{-4}}{\quad\quad\thecontentslabel\quad}{}
        {\titlerule*[10pt]{$\ldots$}\contentspage}
