%%%% 导入文件
\newcommand{\inputfiles}[1]{
    \input{settings/#1}
}
\inputfiles{packages}
\inputfiles{fonts}
\inputfiles{contents}

%%%% 颜色
\definecolor{mygreen}{RGB}{28,172,0} % color values Red, Green, Blue
\definecolor{mylilas}{RGB}{170,55,241}

%%%% 图需要中英文标注
\captionsetup[figure]{font={normalsize,rm},name={图},labelsep=quad} %设置图标题标号后面有一个空格
\captionsetup[subfigure]{font={normalsize,rm},name={图},labelsep=quad} %设置图标题标号后面有一个空格
\captionsetup[figure][bi-second]{name=Figure}   %设置英文图开头为Figure


% 表需要中英文标注
\captionsetup[table][bi-first]{name=表}
\captionsetup[table][bi-second]{name=Table}
\captionsetup[table]{labelsep=space}

%-----------基本页面设置------------------
\setlength{\baselineskip}{20pt}
\setlength{\topmargin}{-0.94cm}  %页眉高度1.5cm-1in
\setlength{\footskip}{0.5552cm}  %页角高度1.75: 下边距2.5-0.0376*10.5-1.75
\pagestyle{empty}
\chead{\zihao{5}江苏科技大学本科毕业设计(论文)}% 使用rhead 定义右侧上方页眉
\rhead{~}% 设置左侧页眉为空
\lhead{~}% 设置右侧页眉为空
\cfoot{\thepage}% 设定页脚为页码
\renewcommand{\headrulewidth}{0.5pt}
\setlength{\headsep}{0.624cm}


%%%%%%%%%%%%%%%%
\numberwithin{equation}{chapter}
\theoremstyle{plain}
\newtheorem{proposition}{命题}[chapter]
\newtheorem{theorem}{定理}[chapter]
\newtheorem{corollary}{结论}[chapter]
\newtheorem{lemma}{引理}[chapter]
\newtheorem{remark}{注释}[chapter]
\newtheorem{definition}{定义}[chapter]
\newtheorem{assumption}{假设}[chapter]
%%%%%%%%%%%%%%%%%%%%%%%%%%
\DeclareMathOperator{\argmin}{argmin}
\DeclareMathOperator{\supp}{supp}
\DeclareMathOperator{\prox}{prox}
\DeclareMathOperator{\sign}{sign}