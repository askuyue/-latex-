	{\centering\chapter{加速割平面法的技巧探究}}

\section{具体求解整数规划例题}
求解如下的整数线性规划:
	\begin{equation}\label{Chapter2ExamILP}
		\begin{split}
			&\max_{x \in \mathbf{R}^{2}}~z = 2 x_1  + 3 x_2  \\
			& \quad \text{s.t.} ~
				\begin{cases}
					3x_1 + 5x_2 &\leq 15; \\
					3x_1 + x_2 &\leq 6;\\
					x_1, x_2 \geq 0, &x_1, x_2 \text{为整数} 
				\end{cases}
		\end{split}
	\end{equation}

求解的过程中发现,增添的“割平面”不能有效地提高得到最优解的速度,而且目标函数值也下降的非常缓慢。
针对以上描述的具体问题,要求同学们按照已有的知识积累,请提出一个解决上述问题的改进方法,找出或者时尝试性地给出相应的改进技巧,来加速“割平面法”的求解速度。

\subsection{分支定界法求解整数规划}

求解如下的整数线性规划:
	\begin{equation*} 
		\begin{split}
			&\max_{x \in \mathbf{R}^{2}}~z = 2 x_1  + 3 x_2  \\
			& \quad \text{s.t.} ~
				\begin{cases}
					3x_1 + 5x_2 &\leq 15; \\
					3x_1 + x_2 &\leq 6;\\
					x_1, x_2 \geq 0, &x_1, x_2 \text{为整数} 
				\end{cases}
		\end{split}
	\end{equation*}

求解的过程中发现,增添的“割平面”不能有效地提高得到最优解的速度,而且目标函数值也下降的非常缓慢。
针对以上描述的具体问题,要求同学们按照已有的知识积累,请提出一个解决上述问题的改进方法,找出或者时尝试性地给出相应的改进技巧,来加速“割平面法”的求解速度。

\subsection{割平面法求解整数规划}

第一步,使用单纯形法求解松弛问题
检测LP的初始单纯形表1.1
表1.1  LP的初始单纯形表
 
 虽然初始单纯形法表的检验数都是非负的,但是结果不是期望的,按照字典顺序法选择入基变量 ,计算 ,因此选择出基变量 
 

在最后的单纯形表
 
把上述左边的变量系数都分解为带有真分数形式,即
 
第二步,寻找Gomory约束,应该选取切割条件较强的约束
若选取 ,则对应这一行的约束为
 
对应的Gomory约束为          
 
若选取 ,则对应这一行的约束为
 
对应的Gomory约束为


\section{探讨分支定界法求解整数规划}

根据上述例题综,我们从分支定界法和割平面法两种方法的实质和例题解题步骤、适用对象等方面做了介绍,从中我们发现运用分支定界法是非常灵活且便于计算机求解的,但是同样,我们也发现,分支定界法的缺点是必须在每一个节点解一个完整的线性规划问题,假使我们的题目是一个大型问题,那么使用分支定界法求解那将是一个十分耗时的工程,且并不是每一个整数规划都能用分支定界法来求解。下面来看一个例子:
  
首先我们先求该整数规划问题的松弛问题,即
 
根据单纯形法求得该松弛问题的最优解为  ,此时的最大值  ,也就是它的初始上界为  ,但是很明显,所得到的最优解并不满足整数条件,因此我们应该继续确定 的的下界 ,假使容易求得一个整数解,那么就可以把这个值作为初始的下界,反之则可以令初始下界为  或者式等到使用分支定界法给出一个整数可行解之后再给出,它的作用就是解的目的仅在于求得比该下界更好的目标值,在此例子中,我们有  .

接下来我们就需要运用分支定界法来求解该问题,首先我们分出各支并求解相应各支的解,我们可以使用图解法便于观看,我们先选取较小的解,即取 ,根据分支定界法使用条件,我们要在原问题上增加两个约束,即要分别增添  形成两个分支,然后我们再分别求解这两个分支的解,当然在求解的过程中我们可能会遇到一些情况:

假使该分支没有可行解,则我们不再对其分支;假使在该分支的最优解得到了整数解,则该分支停止;假使该分支得到非整数解,则要继续分析,如果得到的 值比初始下界还要小,则我们没有必要再对其分支,反之继续分支。当我们遇到这一对分支都需要继续向下分解的情况,那么假如我们求解的目标值是求其极大值,那么我们先将分支中目标函数较小的那一分支暂时搁置,然后分解另一目标函数值较大的分支,直到求解完毕,然后再将留待处理的分支按照“后进先出”的原则对其依次分解求值。

需要注意的是在每次分支求解时,修改原来的上、下界,我们可以每当求出一个新的整数解,就将其对应的目标函数值与之前的进行对比,取最大的,而下界也应在不断的求解中增大(在求目标值为极大值的情况下)。

该例题的分支过程图如下:

\section{加速割平面法的技巧探究}

平面法不断地增加线性约束条件(几何术语称为割平面),从而将原规划问题地可行域割掉一部分,使得切割掉的部分只包含有非整数解,而并没有切割任何整数可行解,就这样不断切割知道得到地可行域有一个整数坐标地极点,它恰好是所求问题的最优解,割平面法对于求解整数规划问题是很有效的,但是从前面的计算过程我们可以发现,同一张单纯形表可以产生不同的切割不等式,那么哪一个不等式的切割效果更好呢?这是一个值得探讨的问题。

我们无法根据某些例题来武断的评判分支定界法与割平面法谁优谁劣,两种方法各有千秋,他们的共同点都是对于可行域切割,刨除掉一部分非整数解域,使得剩下的子域中包含原来整数规划问题的所有可行域,而他们的不同点就在于,割平面法切割剩下的子域仍为一个,而分支定界则是两枝。割平面法对于问题的结构以及求解结果有着较高的要求,也就是要区分纯整数和混合整数问题,不同问题有不同的切割方程,在切割后剩下的仍然是一个子域,后续的计算是一个较小的子域上的整数规划问题,而分支定界法在作分割之后分成了两个分支,后续的计算就是小哥比较小的子域上面的整数规划问题,但是随着分支增多,计算量也越来越大,我们计算时要具体问题具体分析,而下面我们来着重研究一下割平面法。


\subsection{研究常用割平面法优化技巧}

在我们之前的运筹学学习中,我们已经知道了一种优化割平面法选取约束方程的方法,即选取单纯性表中具有最大小(分)数部分的非整分量所在行来构造Gomory约束,那么为什么会有这个方法呢?下面我们来看:
在我们使用割平面法求解整数规划问题时:


使用割平面法来求解上述整数规划问题,首先就需要来求解其松弛问题的最优解,也就是说,我们先不去考虑上述问题中的整数约束条件,直接使用单纯形法求解线性规划问题,得到问题的最优解,假使我们得到了整数最优解,则问题得到解决,所求得的最优解就是原整数规划问题的最优解;然而我们在计算中,松弛问题的最优解往往不能满足其整数约束条件,那么我们就需要增加割平面,几何上来说就是将线性规划可行域切割掉一部分,被切割下去的部分不包括任何整数可行解。之后就是在缩小的可行域上继续求解线性规划问题最优解,我们通过不断地增加割平面方程,使得可行域不断缩小,直到找到原整数规划问题的最优解,那么我们研究的重点也就是如何寻找适当的割平面方程。
下面我们来考虑纯整数规划问题 


\subsection{割平面法改进方案}

在使用割平面法求解纯整数规划问题的时候,我们使用传统的选取割平面方法,即选取非整分量分数部分最大的基变量所在行,但是我们也会遇到一些特殊的情况,比如得到的单纯形表中存在不只一个最大非整分量分数,那么这个情况下,一般来说我们会任意选取一个来做,但是并不清楚哪个割平面方程的约束性更强,就比如我们课题给出的题目,常常切割次数很多,运算量太大,难以快速的找到整数规划最优解。而我们从表3中可以看到割平面选取与 有关,那么将多个最大非整分量分数所在行提取出来列成割平面方程,方程右式分数部分相同,那么我们就可以观察方程左式,我们不妨假设选取最大左式中的系数相加提取其分数部分所在割平面方程,也就是在右式相同的情况下来比较左式系数相加的绝对值中的分数部分,值越大,其所在割平面方程的切割条件越强,下面是运用该改进方法求解整数规划的算法流程图:



