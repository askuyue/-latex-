	% **********绪论**************
	{\centering\chapter{绪论}}
	\setlength{\baselineskip}{20pt}
%%%%%%%%%%%%%%%%%%%%%%%%%%%%%%%%%%%%%%%%%%%%%%%%%%%%

	
	\section{课题背景}
	
线性规划\cite{yangchang2014integer}(Linear Prigramming 简记为LP)是运筹学数学规划中研究较早、发展较快、应用广泛的一个非常重要的分支。线性规划,顾名思义,是由两个关键部分构成,也就是线性与规划,线性是如果一个函数  满足下面可加性与齐次性这两个条件,那么则说明该函数是线性的。
  
  
我们也经常地把一阶多项式函数  称为线性地,其中 是常数。
线性规划自从1947年由G. B. Dantzig提出了解决线性规划问题的单纯形法之后,线性规划问题理论不断趋于成熟,在实际应用当中,由于计算机能够处理多个约束条件和决策变量的线性规划问题,线性规划成为现代管理经常运用的方法之一,我们在解决实际问题时,要把问题转化成为一个线性规划数学模型,我们需要选取恰当的决策变量建立模型,一般来说,线性规划问题的标准型为:
	\begin{equation*}
		\begin{split}
			&\max_{x \in \mathbf{R}^{n}}~z = \sum_{j=1}^{n}c_j x_j \\
			& \quad \text{s.t.} ~
				\begin{cases}
					\sum_{i=1}^{n} a_{ij} x_j = b_i, ~&i = 1, 2, \cdots, m; \\
					x_j \geq 0, ~&j = 1, 2, \cdots, n.
				\end{cases}
		\end{split}
	\end{equation*}

线性规划如今在现实生活中的应用极多,比如,在生产方面,为了适应社会不确定的需求计划,需要利用线性规划的理论方法和仿真模拟方法来估算确认产方的生产值和劳动力分配;在车辆调度问题方面,也可以运用线性规划对城市车辆进行调研分析,对于交通堵塞的情况做出优化路线。

	
	\subsection{线性规划问题的概念}
	
	说到整数规划\cite{ywy2022},就会想到线性规划,线性规划表示所有的约束条件与目标函数皆线性,未知数的次数都是一次,线性规划中包含着线性整数规划。线性规划求解问题的基本方法有单纯形法、改进单纯形法、对偶单纯形法等。而整数(线性)规划是线性规划中未知数只能取整数的特例,它在线性规划的基础上增加了整数约束。整数线性规划可以分为以下类别:
	
	(1)纯整数线性规划,
	
	(2)混合整数线性规划,
	
	(3)0-1型整数线性规划;
	
	其中纯整数线性规划是指全部的决策变量都要取整数值的整数线性规划;混合整数线性规划是指决策变量当中有部分取整数值,另一部分可不取整数值的整数线性规划;0-1型整数线性规划是只决策变量取值0或者1的整数线性规划。
	
线性规划实际就是求解连续变量的线性优化问题,相比线性规划,整数规划是求解整数变量的优化问题,我们研究较多的问题是纯整数线性规划和混合整数线性规划(MILP),与线性规划不同的是,整数规划它所强调的是它的决策变量取值需要为整数,而求解线性规划的方法有时不能保证求得的解满足整数规划的整数条件,所以求解整数线性规划需要使用它的对应方法,我们较为常见的就是分支定界法和割平面法。

	
	\subsection{整数规划问题的概念及分类}
	
	分支定界法(Branch and bound),该方法是20世纪60年代由理查德.卡普(Richard Karp)所提出,理查德.卡普还研究过最大网络流问题,发表了“组合问题中的可归约性”(Reducibility among Combinatorial Problems)重要论文。 分支定界法较为灵活,便于使用计算机求解整数规划。方法命名上就知道此方法核心点就是分支和定界,其中分支是将一个问题细分成为若干个子问题,然后逐个的讨论这些子问题;定界是指在分支过多的情况下,我们需要讨论的情况也变得越来越多,这时便需要定界,在满足(1)得到最优解,(2)根据现有的条件能够排除最优解在这个分支当中,二者其一,就可以定界,达到删除无讨论意义的分支,从而讨论有意义分支的目的。
使用分支定界法来求解整数规划的步骤如下:

(1)	求解整数的松弛问题的最优解

若该最优解为整数,则直接得到整数规划最优解
若该最优解不为整数,则进行下个步骤

(2)	分支定界
我们从不满足整数条件的基变量当中任选一个Xi来进行分支,这个分支必须满足xl ≤[xl ] 或xl ≥[xl ] +1中的任意一个,我们把这两个约束条件加入原问题,从而形成了两个互不相容的子问题(即二分法)。

“分支”为求解整数规划最优解创造条件,“定界”可以来提高搜索的效率。

	
	\subsection{分支定界法概念及运用方法}
	
	割平面法(cutting plane method)是1958年由美国的高莫利(R.E.Gomory)所提出,故而又称为Gormory割平面法,是计算整数规划的另一种常用方法,此方法相较于分支定界法,它的计算量要小很多,不需要每次计算都分两种情况来进行讨论,但他们的相同点都是把整数规划问题转化成为线性规划问题来解决。割平面法的基本思路如下:首先并不去考虑整数性约束条件,直接求解对性的线性规划问题,如果求解线性规划问题得到的最优解恰好是整数,则结束计算,此解为整数规划问题的最优解,如果不是整数,则增加割平面,要满足两个条件:(1)从线性规划问题的可行域当中至少割掉非整数最优解;(2)不能割掉任何整数可行域;之后再缩小的可行域当中继续求解线性规划问题,经过有限次切割,在不断缩小的可行域中的一个整数极点上达到整数规划求解的最优解。
使用割平面法来求解整数规划的步骤如下:

(1)	首先不考虑变量的取整约束,将原问题的数学模型进行标准化,使用单纯形法求解线性规划问题,假使求得最优解非整数则转至下一步。

(2)	将一个“切割不等式”添加到整数规划的约束挑花那种,对上述线性规划问题的可行域进行“切割”。

割平面法的关键就是我们要如何找当适当的割平面方程,我们此课题主要研究就是如何优化割平面法求解整数规划问题。

	
 
	\section{国内外研究现状}
	
	运筹学如今在农业、工业、经济和社会问题等领域都有广泛的应用,而运筹学体系也不断地发展,形成运筹学多个分支,比如有数学规划(线性规划、非线性规划、整数规划、目标规划、动态规划、随即规划等)、图论与网络、排队论、存贮论、对策论、决策论、维修更新理论、搜索论、可靠论和质量管理等方面。
线性规划最早是由法国数学家J.-B,-J,傅里叶于1832年提出的想法,之后C.Valle Pogson在1911年也提出了线性规划的数学想法,但都没有引起人们的注意。直到五十年代之后,出现了一大批研究线性规划问题的理论。在1954年C.Lamkey提出了对偶单纯形法,这是对于解决线性规划问题的一个重要方法,同年,S.加斯等人解决了对于线性规划问题中对于参数规划和灵敏度问题的的分析问题,在1956年,A.塔克提出了互补松弛定理由,G.B.Danzik与P.沃尔夫为分解算法提供了理论知识,八十年代印度数学家N.卡玛卡提出了新的多项式时间算法,这对于求解线性规划问题非常有效。

割平面法是由高莫瑞(R.E.Gomory)1958年提出的,故又称Gomory割平面法,经过三十多年的发展,已经不断出现了很多解决整数规划问题的方法。如今人们大多使用分支定界法和割平面法求解整数规划,割平面法相较于分支定界法,计算量要小许多,不用每次都要分两种情况进行讨论,而是用它特有的简便方法进行选择。

目前多数运筹教科书关于割平面的讲解不够深入,关于加快割平面法的速率只有提出在实际解题时,经验得出若从最优单纯形表中选择具有最大小(分)数部分的非整分量所在行构造割平面约束,往往可以提高“切割”效果,减少“切割”次数。

国内对于运筹学的研究始于上个世纪八十年代,钱颂迪、胡运权结合运筹学,运用线性规划、数学模拟方法来解决生产应用中的合理下料、配料问题,并且应用在物料管理方面。李军、张锦运用了线性规划来解决物流运输当中汽车路线择优的问题。郭耀煌、陈唐民结合了线性规划理论和多目标决策理论,重点解决了车辆调配或者物流运输当中最短时间的数学问题。左永林、柳志新运用动态规划理论解决了供料方案的优化研究的问题。

经过半个世纪我国数学快速发展,但还是习惯性使用给定方法计算数学问题,在使用割平面法求解整数规划问题时,只是运用选取最大分数部分的非整分量所在行构造割平面约束,但如果遇到存在多个最大分数部分相同的情况,大多数是任选一个非整分量所在行来构造割平面约束,导致切割次数增加,无法快速找到最优解。

	
	\section{本文主要研究内容}
	
	1、收集和分析相关信息,翻译好论文文献并翻阅关于整数规划割平面法的论文,将与我们研究课题相关的部分划上重点,对于优化割平面法的部分进行着重分析,针对整数规划问题的研究有利于加强我们对割平面法求解整数规划的认识,可以完善有关割平面法使用的研究。
	
2、利用单纯形法求解松弛问题,观察所给课题题目的松弛问题的最优解是否满足整数条件,假使不满足整数条件,则要继续求解该问题,分别使用分支定界法与割平面法进行该课题例题的解答,并列出相应步骤,对比分析两种方法的相同点与不同点。

3、利用书中所提出的方法选取具有最大小(分)数部分的非整分量所在行构造Gomory约束,由于所给题目具有特殊性无法选择最大非整分量,将两个最大小数非整分量所在行都提取出来构造割平面约束,对比两个约束的结果,根据结果同整数之间的差,选择切割条件较强的Gomory约束,观察规律,给出假设。

4、证明所给假设的正确性。


	\section{本文研究方法}

1、文献研究法:本课题的研究需要阅读查阅大量的文献成果,才能总结出现在该论题的研究情况,我们要寻找出以前研究的不足处和避免论文研究内容的重复性。通过上网查阅或者是图书馆借阅有关于研究整数规划割平面法的书籍文献或论文资料,并且对于重要的内容进行提要分析,从而了解掌握所要研究的割平面法的研究现状,进一步全面地、正确地了解掌握割平面法。

2、实例证明法:在论文中,对于各种求解整数规划问题的方法进行对比分析,结合实际对割平面法进行分析,结合运筹学理论分析如今割平面法求解问题de 发展现状,以及割平面法的特征及优劣点,依据已有的科学理论,以及应用割平面法求解相应课题的实例,提出关于优化割平面法的假想,并且将该猜想作用于各种典型题目,对加快割平面法探究的假想进行证明。

 


	\section{本文研究步骤}
	
	1、利用分支定界法求解给定题目的整数规划问题,使用图解法求出松弛问题的最优解,之后进行分支定界求解最优整数解;割平面法求解给定题目的整数规划问题,使用单纯形法求解松弛问题,增加割平面约束,给出每个步骤的单纯形表,比较两种方法的优劣点,着重分析割平面法求解步骤。
	
2、进行文献检索、阅读与总结,了解割平面法相关内容,知晓相关领域有影响的成果及其内容,并且要避免研究的重复性,总结以往论文的不足之处,并且想办法优化,在此基础上提出优化割平面法的合理假设,并将其应用于各类典型整数规划例题中,以此证明该假设是否成立。

3、确定研究的方法,参考多个经典例题,推导出加快割平面法探究的猜想,并落实研究,进行可行性分析,对于该猜想进行证明。

4、得出结论,完成论文。
